\documentclass{beamer}
\usepackage{listings}
\usepackage{color}
\usepackage{tikz}
\usepackage[top=2cm, bottom=2cm, outer=0cm, inner=0cm]{geometry}
\usetheme{Berkeley}
\usecolortheme{spruce}

\title{Perdita della vegetazione a seguito degli incendi che hanno colpito l'Aspromonte durante l'Estate 2021}
\tikz [remember picture,overlay]\node[opacity=0.3,inner sep=0pt] {\includegraphics{raghidi (1).jpg}}
\author{Francesco Lo Muto}
\begin{document}

\maketitle
\AtBeginSection
{	
\begin{frame}
\frametitle{}
\tableofcontents[currentsection,currentsubsection,currentsubsubsection]
\end{frame}
}
\section{Area di studio}
\begin{frame}{Telerilevamento geo-ecologico}
\begin{itemize}
    \item L’area oggetto di studio è il parco nazionale dell'\textbf{Aspromonte} in provicia di Reggio Calabria. 
    \item Il parco si estende per circa 64.153 ettari, nel 2021 entra a far parte della rete mondiale dei geoparchi curata dall'UNESCO
\end{itemize}  
\end{frame}
\section{Scopo del lavoro}
\begin{frame}{Telerilevamento geo-ecologico}
\begin{itemize}
\item mostrare la perdita di vegetazione dopo una serie di incendi boschivi avvenuti all'interno del \textbf{Parco Nazionale dell'Aspromonte} durante l'estate 2021
\item \textbf{analisi multitemporale} dell'area di studio
\item stato di salute e perdita della vegetazione attraverso il calcolo degli indici spettrali \textbf{DVI} e \textbf{NDVI}
\end{itemize}    
\end{frame}
\section{Metodi e materiali}
\begin{frame}{Telerilevamento geo-ecologico}
\begin{itemize}
    \item Immagini satellitari acquisite tramite il satellite \textbf{sentinel-2A} del programma Copernicus
    \item Elaborazione dati tramite il \textbf{software R}
    \includegraphics[width=0.4\textwidth]{sentinel 2.png}
    \includegraphics[width=0.4\textwidth]{r logo.png}
\end{itemize} 
\end{frame}
\section{Interpretazione dati}
\begin{frame}{Telerilevamento geo-ecologico}
\begin{itemize}
    \item area di studio in colori naturali 
    
    \includegraphics[width=0.35\textwidth]{p1.png}
    \includegraphics[width=0.35\textwidth]{p2.png}
    \centering
    \includegraphics[width=0.35\textwidth]{p3.png}
    
\end{itemize}  
\end{frame}
\begin{itemize}
    \item visualizzazione dell'area in falsi colori
    
 \includegraphics[width=0.35\textwidth]{p1a.png}
    \includegraphics[width=0.35\textwidth]{p2a.png}
    \centering
    \includegraphics[width=0.35\textwidth]{p3a.png}
\end{itemize}
\begin{itemize}
    \item Calcolo indici spettrali \textbf{DVI} e \textbf{NDVI}
    \item \textbf{DVI} è dato dalla sottrazione tra la riflettanza \textbf{NIR} e la riflettanza nel \textbf{ROSSO} e ci aiuta a comprendere lo stato di salute delle piante

\includegraphics[width=0.45\textwidth]{dvip1.png}
\includegraphics[width=0.45\textwidth]{dvip2.png}

\includegraphics[width=0.7\textwidth]{dvip3.png}
\item situazione Luglio 2023
 
\end{itemize}
\newpage
\begin{itemize}
    \item \textbf{NDVI} (Normalized Difference Vegetation Index) 
    \begin{equation}
        NDVI =\frac{NIR-RED}{NIR+RED}
    \end{equation}
    \item utile perché è possibile fare confronti tra figure con numero diverso di bit (immagini con risoluzione radiometrica differente).
\lstinputlisting{CODICE.txt}
\end{itemize}
\newpage
\begin{itemize}
    \item \textbf{NDVI} di Maggio e Settembre 2021, Luglio 2023
\includegraphics[width=0.35\textwidth]{ndvip1.png}
\includegraphics[width=0.35\textwidth]{ndvip2.png}
\centering
\includegraphics[width=0.35\textwidth]{ndvip3.png}
    
\end{itemize}
\section{Conclusioni}
\begin{frame}{Telerilevamento geo-ecologico}
\begin{itemize}
    \item Gli incendi che hanno colpito il Parco dell'Aspromonte durante l'estate 2021 hanno avuto un impatto significativo sulla salute delle piante
    \item 2 anni dopo la situazione appare in parziale recupero
\end{itemize}
    
\end{frame}
\newpage
\tikz[remember picture,overlay] \node[opacity=1,inner sep=0pt] at (current page.center){\includegraphics[width=\paperwidth,height=\paperheight]{raghidi MOD.jpg}}
\end{document}
